\documentclass[a4paper]{recipe}
\usepackage[T1]{fontenc}
\usepackage[dutch]{babel}
\usepackage{bookman}

\newcommand{\bsi}[2]{%
  \fontencoding{T1}\fontfamily{pbs}\fontseries{xl}\fontshape{n}%
  \fontsize{#1}{#2}\selectfont}

\renewcommand{\inghead}{\textbf{Ingredienten (voor 2 personen)}:\ }
\renewcommand{\rechead}{\centering\bsi{24pt}{30pt}}

\makeatletter
\renewcommand*\l@subsubsection{\@dottedtocline{3}{3em}{0em}}
\makeatother

\setlength\parindent{0pt}
\setlength\parskip{2ex plus 0.5ex}

\pagestyle{empty}
\begin{document}
\recipe{Tabbouleh}
\ingred{couscous, 150 ml.;
  1 komkommer; 2 tomaten; 1 avocado; 1 zakje gesneden rauwkost;
	1 bos koriander; 2 tenen knoflook; 1/2 beker yoghurt; 1/2 citroen
peper en zout; olijfolie; zakje shoarmakruiden;
\textbf{optioneel:} gerookte kip; peterselie; paprika; rode peper; olijven; gegrilde aubergine of courgette}

Doe een scheut olijfolie bij de de couscous. Wel dit met 150 ml kokend water en laat haar vijf minuten staan. Snijd de tomaten en de helft van de komkommer in blokjes. Knip koriander fijn. Snijd de avocado klein. Meng dit alles door de couscous met peper en zout.

Voor het sausje rasp je de overgebleven komkommer. Knijp de 2 tenen knoflook erdoorheen, yoghurt en citroen erbij en op smaak brengen met peper en zout.
\end{document}
